 \documentclass{beamer}

\usetheme{MagdeburgFIN}
\usefonttheme{structurebold}
\usepackage{graphicx}
\usepackage{float}
\usepackage{url}
\usepackage{pdfpages}


\title{SIMD Acceleration for Index Structures: A Survey}
\author{Marten Wallewein-Eising}
\date{\today}
\institute{Otto von Guericke University, Magdeburg}

\begin{document}

\begin{frame}[plain]
 \titlepage
\end{frame}

\section[Agenda]{}
\begin{frame}
\frametitle{Agenda}
\tableofcontents
\end{frame}

\section{Motivation}
\begin{frame}
\frametitle{Motivation}
	\begin{center}
		\includegraphics[width=0.75\textwidth]{img/big_picture.pdf}
	\end{center}
	\begin{itemize}
		\item What are state-of-the-art index structures?
		\item Which optimizations do they have in common?
	\end{itemize}
\end{frame}

%\section{B$^{+}$- and Radix-Trees}
%\section{Excursion: B$^{+}$-Tree} 
%\begin{frame}
%\frametitle{Excursion: B$^{+}$-Tree}
%\begin{center}%
%	\includegraphics[width=0.5\textwidth]{img/forest.png}
%\end{center}
%\end{frame}
%\begin{frame}
%\frametitle{B$^{+}$-Tree}
%	\begin{center}
%		\includegraphics[width=0.6\textwidth]{img/bplus_tree.png}
%	\end{center}
%	\begin{itemize}
%		\item N-ary tree with often a large number of children per node
%		\item Only leaf nodes contain keys
%		\item Leaf nodes often linked for range based scans
%	\end{itemize}
%\end{frame}
%\begin{frame}
%\frametitle{Radix-Tree}
%	\begin{center}
%	\includegraphics[width=0.6\textwidth]{img/radix_tree.png}
%	\end{center}
%	\begin{itemize}
%		\item Space optimized prefix tree
%		\item Number of children of every inner node is at least the radix $r$
%		\item Each node that is the only child is merged with its parent
%	\end{itemize}
%\end{frame}
\section{SIMD Style Processing}
\begin{frame}
\frametitle{Single Instruction Multiple Data}
\begin{center}
	\includegraphics[width=0.7\textwidth]{img/simd.pdf}
\end{center}
\begin{itemize}
	\item \textbf{\_\_m128i \_mm\_add\_epi32 (\_\_m128i a, \_\_m128i b)} Adds  4 signed 32-bit integers in a to 4 signed 32-bit integers
	in b.
\end{itemize}
\end{frame}
%\begin{frame}
%\frametitle{Horizontal Vectorization}
%\begin{center}
%	\includegraphics[width=0.6\textwidth]{img/horizontal_vectorization.png}
%\end{center}
%\begin{itemize}
%	\item Using SIMD-instructions to compare data items in parallel
%	\item Compare one search key to multiple keys of the index structure
%	\begin{itemize}
%		\item Search key has to be duplicated
%		\item Often all keys of a node are compared against one search key
%	\end{itemize}
%	\vspace*{0.5cm}
%	\item Opposite: Vertical Vectorization
%	\begin{itemize}
%		\item Not useful, since sequential search key storage in main memory is needed
%	\end{itemize}
%\end{itemize}
%\end{frame}
\section{Surveyed Index Structures}

\begin{frame}
\frametitle{Surveyed Index Structures}
\begin{itemize}
	\item Elf
	\item Seg-Tree/Trie
	\item FAST: Fast Architecture Sensitive Tree
	\item VAST: Vector-Advanced and Compressed Structure Tree
	\item ART: Adaptive Radix Tree
\end{itemize}
\end{frame}

\subsection{Elf}
\begin{frame}
	\frametitle{Elf}
	\begin{center}
		\includegraphics[width=0.65\textwidth]{img/elf.pdf}
	\end{center}
	\begin{itemize}
		\item Multi-dimensional index structure for column-wise storage
		\item Prefix-redundancy elimination on distinct column values
		\item Linearisation for optimized memory layout
	\end{itemize}
	\begin{center}
		\tiny [Broneske et al. ICDE’17]
	\end{center}
\end{frame}

\subsection{Seg-Tree/Trie}
\begin{frame}
	\frametitle{Seg-Tree/Trie}
	\begin{center}
		\includegraphics[width=1.0\textwidth]{img/SegTree.pdf}
	\end{center}
	\begin{itemize}
		\item Each node is a k-ary search tree
		\item Each node is linearised to use k-ary search
		\item $k = \frac{\vert SIMD \vert }{\vert Key \vert}$ partitions, $k-1$ separator keys are compared in parallel
		%\item Each node is a k-ary search tree
	\end{itemize}
	\vspace*{\fill}
	\begin{center}
		\tiny [Zeuch et al. EDBT’14]
	\end{center}
\end{frame}

\subsection{FAST}
\begin{frame}
	\frametitle{Fast Architecture Sensitive Tree}
	\begin{center}
		\includegraphics[width=0.5\textwidth]{img/fast.pdf}
	\end{center}
	\begin{itemize}
		\item Based on binary tree
		\item Hierarchical blocking: SIMD, cache line and page blocks
		\item Efficient register, cache line and page usage
	\end{itemize}
	\begin{center}
		\tiny [Kim et al. SIGMOD’10]
	\end{center}
\end{frame}

\subsection{VAST}
\begin{frame}
	\frametitle{Vector-Advanced and Compressed Structure Tree}
	\begin{itemize}
		\item Extension of FAST
		%\item Same hierarchical blocking
		\item Decrease branch misses avoiding conditional branches
		\item Uses key compression
		\begin{itemize}
			\item Lossy compression for inner nodes
			\item Lossfree compression leaf nodes
		\end{itemize}
		\item  Decompression and error correction of lossy compression has less impact compared to the performance increase with SIMD
	\end{itemize}
	\vspace*{\fill}
	\begin{center}
		\tiny [Yamamuro et al. EDBT’12]
	\end{center}
\end{frame}

\subsection{ART}
\begin{frame}
	\frametitle{Adaptive Radix Tree}
	\begin{center}
		\includegraphics[width=0.5\textwidth]{img/art2.pdf}
	\end{center}
	\begin{itemize}
		\item Uses different node types with different number of keys and children
		\item Due to overfill or underfill of nodes, the node type is changed
		\item Reduced space consumption due to lazy expansion and path compression
	\end{itemize}
	\begin{center}
		 \tiny [Leis et al.  ICDE`13]
	\end{center}

\end{frame}

\section{Evaluation}

\begin{frame}
	\frametitle{Evaluation}
	Considered performance criteria for comparing the surveyed index structures:
	\begin{itemize}
		\item Horizontal Vectorization
		\item Minimized key size
		\item Adapted node sizes / types
		\item Decreased branch misses
		\item Exploit cache lines using blocking and alignment
		\item Usage of Compression
		\item Adapt search algorithm for linearized nodes
	\end{itemize}
\end{frame}

\begin{frame}
\frametitle{Evaluation}
Implementation of the considered performance criteria and their impact:
\begin{figure}
	\includegraphics[width=0.9\textwidth]{img/table.pdf}
\end{figure}


\end{frame}

\section{Conclusion}
\begin{frame}
\frametitle{Conclusion}
How to adapt index structures to modern database systems:
\begin{itemize}
	\item Compare as many keys as possible in parallel with SIMD
	\begin{itemize}
		\item Direct performance increase up to a multiple
	\end{itemize}
	\item Efficient usage of cache line
	\item Decrease branch misses
	\item Use compression or/and adapted search algorithms
	\end{itemize}
\end{frame}

\begin{frame}
\frametitle{Conclusion}
How to adapt index structures to modern database systems:
\begin{itemize}
	\item Compare as many keys as possible in parallel with SIMD
	\begin{itemize}
		\item Direct performance increase up to a multiple
	\end{itemize}
	\item Efficient usage of cache line
	\item Decrease branch misses
	\item Use compression or/and adapted search algorithms
\end{itemize}
\vspace*{\fill}
\begin{center}
	\huge Thank you for your attention!
\end{center}
\end{frame}

\begin{frame}
\frametitle{Sources}
\begin{itemize}
	\item \url{http://infolab.stanford.edu/~nsample/cs245/handouts/hw2sol/sol2.html}
	%\item \url{https://en.wikipedia.org/wiki/Radix\_tree}
	\item \url{https://www.clker.com/clipart-bosque.html}
	%\item \url{http://deacademic.com/pictures/dewiki/51/300px-Bplustree.png}
	\item Datenbanken Implementierungstechniken, Ausgabe 3, Saake und Sattler
\end{itemize}
\end{frame}

\begin{frame}
\frametitle{Sources}
\begin{itemize}
	\item T. Yamamuro, M. Onizuka, T. Hitaka, and M. Yamamuro ``VAST-Tree: A Vector-Advanced and Compressed Structure for Massive Data Tree Traversal'' in EDBT, pp. 396-407, 2012.
	\item S. Zeuch, F. Huber and J.-C. Freytag  ``Adapting Tree Structures for Processing with SIMD Instructions'' in EDBT, 2014.
	\item C. Kim, J. Chhugani, N. Satish, E. Sedlar, A. D. Nguyen, T. Kaldewey, V. W. Lee, S. A. Brandt and P. Dubey ``FAST: Fast Architecture Sensitive Tree Search on Modern CPUs and GPUs'' in SIGMOD, pp. 339-350, 2010.
	\item V. Leis, A. Kemper and T. Neumann ``The Adaptive Radix Tree: ARTful Indexing for Main-Memory Databases'' in ICDE, pages 38-49, 2013.
\end{itemize}
\end{frame}

%\begin{frame}
% \frametitle{Thank you for your attention!}
%\end{frame}
\end{document}

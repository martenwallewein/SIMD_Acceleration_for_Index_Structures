\documentclass[conference]{IEEEtran}
\IEEEoverridecommandlockouts
% The preceding line is only needed to identify funding in the first footnote. If that is unneeded, please comment it out.
\usepackage{cite}
\usepackage{amsmath,amssymb,amsfonts}
\usepackage{algorithmic}
\usepackage{graphicx}
\usepackage{textcomp}
\def\BibTeX{{\rm B\kern-.05em{\sc i\kern-.025em b}\kern-.08em
    T\kern-.1667em\lower.7ex\hbox{E}\kern-.125emX}}
\begin{document}

\title{SIMD Acceleration for Index Structures\\
}

\author{\IEEEauthorblockN{Marten Wallewein-Eising}
\IEEEauthorblockA{\textit{Otto-von-Guericke University} \\
Magdeburg, Germany \\
marten.wallewein-eising@st.ovgu.de}
}

\maketitle

\begin{abstract}
\begin{itemize}
	\item summary: 
	\subitem Give short an overview of SIMD and modern index structures
	\subitem Explain what are the problems of the ``old" index structures made for disk-based database systems
	\subitem Explain which approaches were made to adapt index structures to modern systems and what they have in common and what are differences
	\item Why is this work important: 
	\subitem Give a state of current development of the index structures
	\subitem Collect common approaches to adapt other index structures TODO: ReThink
	\item K-ary search trees, FAST, VAST and ART compared
	\item Contribution: What are important approaches used by different implementations to adapt index structures to modern systems
\end{itemize}
\end{abstract}

\begin{IEEEkeywords}
SIMD, index 
\end{IEEEkeywords}

\section{Introduction}
\begin{itemize}
	\item General Problem: Index structures are not applicable to modern systems
	\subitem IO-Bottleneck moves from disk-ram  to ram-cache. CPU cycles of pure calculation got more important, cache line has to be used in an optimized way
	\subitem Index structures grow very high because of the massive amount of data collected in modern databases
	\subitem Branch-mispredictions cost many CPU cycles and should therefore be avoided
	\item objectives/contributions: Comparision of current work and highlighting of important approaches for index structures
	\item main result: Adapting of tree nodes to cache line and SIMD blocks and use small keys to compare as much keys as possible with one SIMD instruction

\end{itemize}
After decades of creating and improving index structures for disk-based database systems, nowadays even large databases fit into the main memory. Since index structures like the $B^+$-tree [TODO: ref] or the radix tree [TODO: ref] have an important part in database-systems to realise SCAN or range-based search operations, these index structures experienced many adaptions to fulfill the needs of modern database-systems. Instead of overcoming the bottleneck of IO-oprations from disk to RAM, the target of modern index-structures is to improve the usage of CPU cache and processor architectures.



\section{Preliminaries}

\subsection{Single instruction multiple data}\label{SCM}
Short overview of the following:
\begin{itemize}
	\item How SIMD works
	\item SIMD vs vertical vector processing
\end{itemize}
\subsection{Considered index structures}\label{SCM}
Short overview of the following: TODO: Maybe the problems of these old index structures?
\begin{itemize}
	\item Binary tree???
	\item $B^{+}$ tree
	\item Radix tree
\end{itemize}

\section{Adapted Tree Structures}
TODO: Compare all 4 or merge FAST and VAST together/ extend FAST with VAST??
\subsection{K-ary search tree}\label{SCM}

\subsection{FAST}\label{SCM}

\subsection{ART}\label{SCM}

\subsection{VAST}\label{SCM}

\section{Evaluation}
In common:
\begin{itemize}
	\item SIMD instructions used to compare the search key with multiple keys of the index
	\item Segmenting tree to blocks for a better usage of cache lines, save the data of the nodes in an adapted way
	\item The keys should be as short as possible to compare more keys in one step and to decrease the passed data to the cache line
	\item Each approach improves the tree traversal
\end{itemize}

Differences:
\begin{itemize}
	\item Node compression in VAST, Path compression in ART and K-ary seg trie
	\item FAST and K-ary trees readonly to improve traversal, ART and FAST adapt insert too
	\item FAST uses and K-ary trees will use GPU calculation instead of CPU
\end{itemize}

Why performance can not be compared in a useful way...

\section{Related Work}
TODO:
\begin{itemize}
	\item ART and VAST compared to FAST??
	\item Ideas and implementations of the adapted trees already in III...
	\item KD-Tree with SIMD
\end{itemize}
%TODO: Add related work:
%\begin{itemize}
%	\item GPU and SIMD
%	\item Other tree structures, T-Tree, CSB-Tree, HAT-Tree, KISS-Tree
%\end{itemize}

%\begin{itemize}
%	\item MainMemory Index Structures with Fixed Size Partial Keys, looking at tries
%	\item Rethinking SIMD Vectorization for In-Memory Databases
%	\item Implementing Database Operations Using SIMD Instructions, comparison of SIMD vs vector processing, other db operations Ref: D. R. Morrison, “PATRICIA-practical algorithm to retrieve information coded in alphanumeric,” J. ACM, vol. 15, no. 4, 1968.
%	\item k-ary search: B. Schlegel, R. Gemulla, and W. Lehner, “k-ary search on modern	processors,” in DaMoN workshop, 2009.
%	\item ART and VAST compare to FAST
%	\item better cache line behaviour of B+ trees Ref: J. Rao and K. A. Ross. Cache conscious indexing for decision support in main	memory. In VLDB, pages 78–89, 1999.
%	\item Efficient cache line usage with compression Ref: M. Zukowski, S. Heman, N. Nes, and P. Boncz. Super-scalar ram-cpu cache	compression. In ICDE, page 59, 2006.
%\end{itemize}
\section{Conclusion}

\section{Future work}
Open questions
\begin{thebibliography}{00}
\bibitem{b1}Mohammad Suaib, Abel Palaty and Kumar Sambhav Pandey, ``Architecture of SIMD Type Vector Processor'' in International Journal of Computer Applications (0975 - 8887) Volume 20 No.4, April 2011.
\bibitem{b2} Jingren Zhou and Kenneth A. Ross  ``Implementing Database Operations Using SIMD Instructions'' in ACM S1GMOD '2002 June 4-6, Madison, Wisconsin, USA
\bibitem{b3} Steffen Zeuch, Frank Huber and Johann-Christoph Freytag  ``Adapting Tree Structures for Processing with SIMD Instructions'' in Proc. 17th International Conference on Extending Database Technology (EDBT), March 24-28, 2014, Athens, Greece
\bibitem{b4} Viktor Leis, Alfons Kemper and Thomas Neumann ``The Adaptive Radix Tree: ARTful Indexing for Main-Memory Databases'' in ???
\bibitem{b5} Takeshi Yamamuro, Makoto Onizuka,Toshio Hitaka, and Masashi Yamamuro ``VAST-Tree: A Vector-Advanced and Compressed Structure for Massive Data Tree Traversal'' in EDBT 2012, March 26-30, 2012, Berlin, Germany.
\bibitem{b6} Changkyu Kim, Jatin Chhugani, Nadathur Satish, Eric Sedlar, Anthony D. Nguyen,
Tim Kaldewey, Victor W. Lee, Scott A. Brandt and Pradeep Dubey ``FAST: Fast Architecture Sensitive Tree Search
on Modern CPUs and GPUs'' in SIGMOD’10, June 6-11, 2010, Indianapolis, Indiana, USA.
\end{thebibliography}

\end{document}

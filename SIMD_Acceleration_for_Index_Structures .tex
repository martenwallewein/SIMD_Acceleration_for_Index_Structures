\documentclass[conference]{IEEEtran}
\IEEEoverridecommandlockouts
% The preceding line is only needed to identify funding in the first footnote. If that is unneeded, please comment it out.
\usepackage{cite}
\usepackage{amsmath,amssymb,amsfonts}
\usepackage{algorithmic}
\usepackage{graphicx}
\usepackage{textcomp}
\def\BibTeX{{\rm B\kern-.05em{\sc i\kern-.025em b}\kern-.08em
    T\kern-.1667em\lower.7ex\hbox{E}\kern-.125emX}}
\begin{document}

\title{SIMD Acceleration for Index Structures\\
}

\author{\IEEEauthorblockN{Marten Wallewein-Eising}
\IEEEauthorblockA{\textit{Otto-von-Guericke University} \\
Magdeburg, Germany \\
marten.wallewein-eising@st.ovgu.de}
}

\maketitle

\begin{abstract}
\begin{itemize}
	\item summary: 
	\subitem Give short an overview of SIMD and modern index structures
	\subitem Explain what are the problems of the ``old" index structures made for disk-based database systems
	\subitem Explain which approaches were made to adapt index structures to modern systems and what they have in common and what are differences
	\item Why is this work important: 
	\subitem Give a state of current development of the index structures
	\subitem Collect common approaches to adapt other index structures TODO: ReThink
	\item K-ary search trees, FAST, VAST and ART compared
	\item Contribution: What are important approaches used by different implementations to adapt index structures to modern systems
\end{itemize}
\end{abstract}

\begin{IEEEkeywords}
SIMD, index 
\end{IEEEkeywords}

\section{Introduction}
% Main Problem: Index structures designed for disk-based database systems
After decades of creating and improving index structures for disk-based database systems, nowadays even large databases fit into the main memory. Since index structures like the $B^+$-tree [TODO: ref] or the radix tree [TODO: ref] have an important part in database-systems to realise SCAN or range-based search operations, these index structures experienced many adaptions to fulfill the needs of modern database-systems. Instead of overcoming the bottleneck of IO-oprations from disk to RAM, the target of modern index-structures is to improve the usage of CPU cache and processor architectures.

% Introduce tree structures and specific Problem: Cache and TLB misses and branch mispredictions cost a lot of calculation effort
In this paper we compare different approaches to adapt index structures to use Single Instruction Multiple Data (SIMD) [1] operations and to better overcome the bottleneck from RAM to CPU cache. We consider the K-ary Search Tree (Seg-tree) [TODO: ref], Adapted Radix Tree (ART) [4], Fast Architecture Sensitive Tree (FAST) [6], and Vector-Advanced and Compressed Structure Tree (VAST) [5]. As the authors of VAST-Tree show, important causes for increased calculation time are cache misses and branch mispredictions. To overcome branch mispredictions and to decrease CPU cycles, SIMD [TODO: ref] is used in modern index structures for tree traversal. The authors of the k-ary search show how to use SIMD to compare mutliple keys in one CPU cycle. To decrease cache misses, the authors of FAST and ART show how to adapt index structures to the cache line size.  

% Objectives and Contribution
All approaches use SIMD only for key comparison within tree traversal and try to decrease the key size to fit more keys into one SIMD register. Therefore FAST and Seg-tree only provide implementations for search algorithms. We consider the approaches VAST and ART make to implement operations like update and insert und name ideas to use SIMD for them. Consequently, with this work we make the following contributions:
\begin{itemize}
	\item We compare different adaptions of index structures to fulfill requirements of modern database systems
	\item We highlight the usage of SIMD and the cache line adaptions in all approaches
	\item We show openings for adaption for other approaches to use SIMD
\end{itemize}
% Paper structure
We organized the rest of the paper as follows. In section 2 we give the preliminaries for SIMD in general and for the use in index structures. In section 3 we analyse the different approaches of adapted index structures and evaluate the comparision in section 4. In section 5 we name related work. In section 5 we present our conclusion und describe future work in section 7. 
\section{Preliminaries}
% Horizontal vs vertical vector processing
A common approach of decreasing CPU cycles for algorithms is to adapt the algorithm to pipelining. While one instruction is executed, the next instruction is already fetched. In contrast to execute one operation on one date after another, the idea of SIMD is to execute a single instruction on multiple data. The authors of Rethinking SIMD Vectorization for In-Memory Databases [TODO: ref] show two general approaches to use SIMD in in-memory databases, horizontal and vertical vector processing. They name the comparision of one search key to multiple other keys horizontal vectorization, whereas processing a different input key per vector lane is named vertical vectorization. TODO: Exact difference?

% How horizontal vectorization works
Since FAST, Seg-Tree, ART and VAST only use horizontal vectorization, we focus on this approach. Modern CPUs have additional SIMD registers along with an additional instruction set adapted to process multiple data with a single instruction. For example, the authors of Seg-Tree [3] use 128-bit SIMD registers and adjusted SIMD operations to load data into a register and to compare the data of one SIMD register with another. A 128-bit SIMD register processes sixteen 8-bit or eight 16-bit data items with one instruction.
\section{Adapted Tree Structures}
TODO: Compare all 4 or merge FAST and VAST together/ extend FAST with VAST??
\subsection{K-ary search tree}\label{SCM}

\subsection{FAST}\label{SCM}

\subsection{ART}\label{SCM}

\subsection{VAST}\label{SCM}

\section{Evaluation}
In common:
\begin{itemize}
	\item SIMD instructions used to compare the search key with multiple keys of the index
	\item Segmenting tree to blocks for a better usage of cache lines, save the data of the nodes in an adapted way
	\item The keys should be as short as possible to compare more keys in one step and to decrease the passed data to the cache line
	\item Each approach improves the tree traversal
\end{itemize}

Differences:
\begin{itemize}
	\item Node compression in VAST, Path compression in ART and K-ary seg trie
	\item FAST and K-ary trees readonly to improve traversal, ART and FAST adapt insert too
	\item FAST uses and K-ary trees will use GPU calculation instead of CPU
\end{itemize}

Why performance can not be compared in a useful way...

\section{Related Work}
TODO:
\begin{itemize}
	\item ART and VAST compared to FAST??
	\item Ideas and implementations of the adapted trees already in III...
	\item KD-Tree with SIMD
\end{itemize}

\section{Conclusion}

\section{Future work}
Open questions, use SIMD for tree creation/updates instead of only for traversal
\begin{thebibliography}{00}
\bibitem{b1}Mohammad Suaib, Abel Palaty and Kumar Sambhav Pandey, ``Architecture of SIMD Type Vector Processor'' in International Journal of Computer Applications (0975 - 8887) Volume 20 No.4, April 2011.
\bibitem{b2} Jingren Zhou and Kenneth A. Ross  ``Implementing Database Operations Using SIMD Instructions'' in ACM S1GMOD '2002 June 4-6, Madison, Wisconsin, USA
\bibitem{b3} Steffen Zeuch, Frank Huber and Johann-Christoph Freytag  ``Adapting Tree Structures for Processing with SIMD Instructions'' in Proc. 17th International Conference on Extending Database Technology (EDBT), March 24-28, 2014, Athens, Greece
\bibitem{b4} Viktor Leis, Alfons Kemper and Thomas Neumann ``The Adaptive Radix Tree: ARTful Indexing for Main-Memory Databases''
\bibitem{b5} Takeshi Yamamuro, Makoto Onizuka,Toshio Hitaka, and Masashi Yamamuro ``VAST-Tree: A Vector-Advanced and Compressed Structure for Massive Data Tree Traversal'' in EDBT 2012, March 26-30, 2012, Berlin, Germany.
\bibitem{b6} Changkyu Kim, Jatin Chhugani, Nadathur Satish, Eric Sedlar, Anthony D. Nguyen,
Tim Kaldewey, Victor W. Lee, Scott A. Brandt and Pradeep Dubey ``FAST: Fast Architecture Sensitive Tree Search
on Modern CPUs and GPUs'' in SIGMOD’10, June 6-11, 2010, Indianapolis, Indiana, USA.
\end{thebibliography}

\end{document}
